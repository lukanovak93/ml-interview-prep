\subsection{Three friends in Seattle told you it's rainy. Each has probability of 1/3 of lying. What's the probability of Seattle is rainy?}
\subsection{Discuss how to randomly select a sample from a product user population.}
\subsection{Q: Define variance.}
\bigskip
Variance measures how far a dataset is spread out. The technical definition is "\textit{The average of the squared differences from the mean}", but all it really does is to give you a very general idea of the spread of your data. A value of zero means that there is no variability, in fact, all the numbers in the dataset are the same.\par
\bigskip
The variance (${\sigma}^2$) is a measure of how far each value in the data set is from the mean. Here is how it is defined: 
Subtract the mean from each value in the data. This gives you a measure of the distance of each value from the mean.
Square each of these distances (so that they are all positive values), and add all of the squares together.
Divide the sum of the squares by the number of values in the data set.
The standard deviation ($\sigma$) is simply the (positive) square root of the variance. 
\bigskip
\[
{\sigma}^2 = \frac{\sum_{i=1}^{N} {(X - \mu)}^2}{N}
\]
where $X$ is the \textbf{mean value} and $\mu$ is \textbf{current example value}
\bigskip
\subsection{How do you find percentile? Write the code for it.}
The most common definition of a \textit{percentile} is a number where a certain percentage of scores fall below that number. $$Percentile = (number of people behind you / total number of people) * 100$$

\begin{lstlisting}[language=python]
def percentile(p, values):
    r = round((p/100)*len(values))
    range_of_percentile = values[:r]
    percentile_value = values[r]
    return(range_of_percentile, percentile_value)

if __name__ == '__main__':
    values = list(range(100))
    print(percentile(30, values))
\end{lstlisting}
 
\begin{lstlisting}
>>> ([0, 1, 2, 3, ... , 27, 28, 29], 30)
\end{lstlisting}